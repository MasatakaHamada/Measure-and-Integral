\documentclass{jsreport}
\usepackage{amsmath, amssymb} % 数学記号のパッケージ

% 定理などの番号付けに関するパッケージとその設定
\usepackage{amsthm}
\theoremstyle{definition}
\newtheorem{defi}{定義}[section]
\newtheorem{lem}[defi]{補題}
\newtheorem{prop}[defi]{命題}
\newtheorem{thm}[defi]{定理}
\newtheorem{cor}[defi]{系}
\newtheorem{rem}[defi]{注意}
\newtheorem{exq}[defi]{例題}
\newtheorem{qst}[defi]{問}

\title{抽象論から始めるルベーグ積分入門}
\author{}

\begin{document}
\maketitle
\tableofcontents

\part{一般の測度空間における積分}

\chapter{測度空間}

\section{有限加法族}

\begin{defi}\label{def_algebra}
集合$S$に対し,$S$の部分集合族$\Sigma_0$が$S$上の\textbf{有限加法族}であるとは,
次の条件をすべて満たすことである:
\begin{enumerate}
\item$S\in\Sigma_0$
\item$A\in\Sigma_0 \Rightarrow A^c\in\Sigma_0$
\item$A_1,A_2\in\Sigma_0 \Rightarrow A_1 \cup A_2\in\Sigma_0$
\end{enumerate}
\end{defi}

\begin{qst}\label{qst_algebra}
定義\ref{def_algebra}の条件
$A_1,A_2\in\Sigma_0 \Rightarrow A_1 \cup A_2\in\Sigma_0$
は次の条件と同値であることを示せ:
任意の$n=1,2,\cdots$に対して,
$A_1,A_2,\cdots,A_n\in\Sigma_0 \Rightarrow A_1 \cup A_2 \cup\cdots\cup A_n\in\Sigma_0$
\end{qst}

\section{$\sigma$-加法族}

\begin{defi}\label{def_sigma_algebra}
集合$S$に対し,$S$の部分集合族$\Sigma$が$S$上の\textbf{$\sigma$-加法族}であるとは,
次の条件をすべて満たすことである:
\begin{enumerate}
\item$S\in\Sigma$
\item$A\in\Sigma \Rightarrow A^c\in\Sigma$
\item$A_1,A_2,\cdots\in\Sigma \Rightarrow \displaystyle \bigcup_{n=1}^\infty A_n\in\Sigma$
\end{enumerate}
\end{defi}

\begin{qst}\label{qst_sigma_algebra_is_algebra}
集合$S$上の$\sigma$-加法族は$S$上の有限加法族であることを示せ.
\end{qst}

\begin{qst}\label{qst_countable_intersection}
集合$S$上の$\sigma$-加法族$\Sigma$に対して,次が成り立つことを示せ:
$A_1,A_2,\cdots\in\Sigma \Rightarrow \displaystyle \bigcap_{n=1}^\infty A_n\in\Sigma$
\end{qst}

\begin{prop}\label{prop_minimal_sigma_algebra}
集合$S$の任意の部分集合族$\mathcal{A}$に対し,$\mathcal{A}$を含む最小の$\sigma$-加法族が存在する.
すなわち,次の条件をすべて満たす$\sigma$-加法族$\Sigma$が存在する:
\begin{enumerate}
\item$\mathcal{A}\subseteq\Sigma$
\item$\mathcal{A}\subseteq\Sigma'$を満たす任意の$\sigma$-加法族$\Sigma'$に対して$\Sigma\subseteq\Sigma'$
\end{enumerate}
\end{prop}

\begin{defi}\label{def_minimal_sigma_algebra}
集合$S$の任意の部分集合族$\mathcal{A}$に対し,
$\mathcal{A}$を含む最小の$\sigma$-加法族を$\sigma(\mathcal{A})$と書く.
\end{defi}

\begin{defi}\label{def_Borel_set}
位相空間$S$に対し,$S$の開集合をすべて含む最小の$\sigma$-加法族を$\mathcal{B}(S)$と書く.
\end{defi}

\section{可測空間}

\begin{defi}\label{def_measurable_space}
集合$S$と$S$上の$\sigma$-加法族$\Sigma$の組$(S,\Sigma)$を\textbf{可測空間}と呼ぶ.
\end{defi}

\section{有限加法性(仮)}

\begin{defi}\label{def_disjoint_union}
集合$A_1,A_2,\cdots$が,$i \neq j$に対して$A_i \cap A_j=\varnothing$を満たすとき,
\[ \sum_{n=1}^k A_n:=\bigcup_{n=1}^k A_n,\sum_{n=1}^\infty A_n:=\bigcup_{n=1}^\infty A_n \]
と定める.
\end{defi}

\section{測度}

\begin{defi}\label{def_measure}
可測空間$(S,\Sigma)$上の関数$\mu \colon \Sigma\to[0,\infty]$が\textbf{測度}であるとは,
次の条件をすべて満たすことである:
\begin{enumerate}
\item$\mu(\varnothing)=0$
\item$\displaystyle\mu\left(\sum_{n=1}^\infty A_n\right)=\sum_{n=1}^\infty\mu(A_n)$
\end{enumerate}
\end{defi}

\section{測度空間}

\begin{defi}\label{def_measure_space}
可測空間$(S,\Sigma)$および$(S,\Sigma)$上の測度$\mu \colon \Sigma\to[0,\infty]$に対して,
組$(S,\Sigma,\mu)$を\textbf{測度空間}と呼ぶ.
\end{defi}

\chapter{可測関数}

$(S,\Sigma)$を可測空間とする.

\section{可測関数}

\begin{defi}\label{def_measurable_function}
$S$上の関数$h \colon S\to\mathbb{R}$が\textbf{$\Sigma$-可測関数}であるとは,
任意の$A\in\Sigma$に対して$h^{-1}(A)\in\mathcal{B}(\mathbb{R})$が成り立つことをいう.
\end{defi}

\begin{defi}\label{def_indicator_function}
$A\in\Sigma$に対し,$S$上の関数$1_A \colon S\to\mathbb{R}$を
\[ 1_A(s):=\begin{cases}1 & (s \in A) \\ 0 & (s \notin A)\end{cases} \]
により定める.このような関数を\textbf{定義関数}と呼ぶ.
\end{defi}

\begin{qst}\label{qst_indicator_function_is_measurable}
$A\in\Sigma$に対し,定義関数$1_A \colon S\to\mathbb{R}$は$\Sigma$-可測関数であることを示せ.
\end{qst}

\begin{defi}\label{def_positive_and_negative_part}
関数$f \colon S\to\mathbb{R}$に対して
$f^+(s):=\max\{f(s),0\}, f^-(s):=\max\{-f(s),0\}$
と定める.
\end{defi}

\begin{prop}\label{prop_positive_and_negative_part_are_measurable}
$\Sigma$-可測関数$f \colon S\to\mathbb{R}$に対して,$f^+,f^-$は$\Sigma$-可測関数である.
\end{prop}

\begin{prop}\label{prop_measurable_function_and_arithmetic}
$\Sigma$-可測関数$h_1,h_2 \colon S\to\mathbb{R}$および$a\in\mathbb{R}$に対し,
$h_1+h_2, ah_1, h_1h_2$は$\Sigma$-可測関数である.
\end{prop}

\begin{defi}\label{def_simple_function}
$a_1,a_2,\cdots,a_n\in\mathbb{R}$および$A_1,A_2,\cdots,A_n\in\Sigma$に対し,
\[ \sum_{i=1}^n a_i1_{A_i}(s) \]
の形で表せる$S$上の関数を\textbf{単関数}と呼ぶ.
\end{defi}

\chapter{積分}

$(S,\Sigma,\mu)$を測度空間とする.

\section{定義関数の積分}

\begin{defi}\label{def_indicator_function_integral}
$A\in\Sigma$に対し,定義関数$1_A \colon S\to\mathbb{R}$の積分を
\[ \int_S 1_A(s)\mu(ds):=\mu(A) \]
で定める.
\end{defi}

\section{単関数の積分}

\begin{defi}\label{def_simple_function_integral}
$a_1,a_2,\cdots,a_n\in\mathbb{R}$および$A_1,A_2,\cdots,A_n\in\Sigma$に対し,
単関数$f(s)=\displaystyle\sum_{i=1}^n a_i1_{A_i}(s)$の積分を
\[ \int_S f(s)\mu(ds):=\sum_{i=1}^n a_i\mu(A_i) \]
で定める.
\end{defi}

\begin{qst}\label{qst_simple_function_integral}
単関数の積分の定義\ref{def_simple_function_integral}はwell-definedであることを示せ.
すなわち,単関数$f \colon S\to\mathbb{R}$が
\[ f(s)=\sum_{i=1}^n a_i1_{A_i}(s)=\sum_{i=1}^m b_i1_{B_i}(s) \]
の2通りで表されるとき,
\[ \sum_{i=1}^n a_i\mu(A_i)=\sum_{i=1}^m b_i\mu(B_i) \]
となることを示せ.
\end{qst}

\section{非負可測関数の積分}

\begin{thm}[\textbf{単調収束定理}]\label{monotone_convergence_theorem}
非負$\Sigma$-可測関数列$\{f_n\}_{n=1}^\infty$と$\Sigma$-可測関数$f$が,
すべての$s \in S$に対して$0 \leq f_1(s) \leq f_2(s) \leq\cdots$かつ
$\displaystyle\lim_{n\to\infty}f_n(s)=f(s)$を満たすとき,次が成り立つ:
\[ \lim_{n\to\infty}\int_S f_n(s)\mu(ds)=\int_S f(s)\mu(ds) \]
\end{thm}

\section{可測関数の積分}

\begin{defi}\label{def_measurable_function_integral}
$\Sigma$-可測関数$f \colon S\to\mathbb{R}$の積分を
\[ \int_S f(s)\mu(ds):=\int_S f^+(s)\mu(ds)-\int_S f^-(s)\mu(ds)  \]
で定める.
\end{defi}

\section{優収束定理}

\begin{thm}[\textbf{ファトゥの補題}]\label{Fatou_lemma}
非負$\Sigma$-可測関数列$\{f_n\}_{n=1}^\infty$に対して次が成り立つ:
\[ \liminf_{n\to\infty}\int_S f_n(s)\mu(ds)\leq\int_S\liminf_{n\to\infty}f_n(s)\mu(ds) \]
\end{thm}

% \part{$\mathbb{R}$上のルベーグ積分}
% \chapter{1次元ルベーグ測度}
% \chapter{ルベーグ可測空間}
% \chapter{可測関数}
% \chapter{積分}

\end{document}