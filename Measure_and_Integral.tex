\documentclass{jsreport}
\usepackage{amsmath, amssymb} % 数学記号のパッケージ

% 定理などの番号付けに関するパッケージとその設定
\usepackage{amsthm}
\theoremstyle{definition}
\newtheorem{defi}{定義}[section]
\newtheorem{lem}[defi]{補題}
\newtheorem{prop}[defi]{命題}
\newtheorem{thm}[defi]{定理}
\newtheorem{cor}[defi]{系}
\newtheorem{rem}[defi]{注意}
\newtheorem{exq}[defi]{例題}
\newtheorem{qst}[defi]{問}
\renewcommand\proofname{\bf 証明} % proof環境の設定変更

% ハイパーリンクに関するパッケージとその設定
\usepackage[dvipdfmx]{hyperref}
\usepackage{pxjahyper}
\hypersetup{
 colorlinks=true,
 linkcolor=[rgb]{0,0,0}
}

\title{抽象論から始めるルベーグ積分入門}
\author{}

\begin{document}
\maketitle
\tableofcontents

\part{一般の測度空間における積分}

\chapter{測度空間}

\section{有限加法族}

\begin{defi}\label{def_algebra}
集合$S$に対し,$S$の部分集合族$\Sigma_0$が$S$上の\textbf{有限加法族}であるとは,
次の条件をすべて満たすことである:
\begin{enumerate}
\item$S\in\Sigma_0$
\item$A\in\Sigma_0 \Rightarrow A^c\in\Sigma_0$
\item$A_1,A_2\in\Sigma_0 \Rightarrow A_1 \cup A_2\in\Sigma_0$
\end{enumerate}
\end{defi}

\begin{qst}\label{qst_algebra}
定義\ref{def_algebra}の条件
$A_1,A_2\in\Sigma_0 \Rightarrow A_1 \cup A_2\in\Sigma_0$
は次の条件と同値であることを示せ:
任意の$n=1,2,\cdots$に対して,
$A_1,A_2,\cdots,A_n\in\Sigma_0 \Rightarrow A_1 \cup A_2 \cup\cdots\cup A_n\in\Sigma_0$
\end{qst}

\section{$\sigma$-加法族}

\begin{defi}\label{def_sigma_algebra}
集合$S$に対し,$S$の部分集合族$\Sigma$が$S$上の\textbf{$\sigma$-加法族}であるとは,
次の条件をすべて満たすことである:
\begin{enumerate}
\item$S\in\Sigma$
\item$A\in\Sigma \Rightarrow A^c\in\Sigma$
\item$A_1,A_2,\cdots\in\Sigma \Rightarrow \displaystyle \bigcup_{n=1}^\infty A_n\in\Sigma$
\end{enumerate}
\end{defi}

\begin{qst}\label{qst_sigma_algebra_is_algebra}
集合$S$上の$\sigma$-加法族は$S$上の有限加法族であることを示せ.
\end{qst}

\begin{qst}\label{qst_countable_intersection}
集合$S$上の$\sigma$-加法族$\Sigma$に対して,次が成り立つことを示せ:
$A_1,A_2,\cdots\in\Sigma \Rightarrow \displaystyle \bigcap_{n=1}^\infty A_n\in\Sigma$
\end{qst}

\begin{prop}\label{prop_generated_sigma_algebra}
集合$S$の任意の部分集合族$\mathcal{A}$に対し,$\mathcal{A}$を含む最小の$\sigma$-加法族が存在する.
すなわち,次の条件をすべて満たす$S$上の$\sigma$-加法族$\Sigma$が存在する:
\begin{enumerate}
\item$\mathcal{A}\subseteq\Sigma$
\item$\mathcal{A}\subseteq\Sigma'$を満たす任意の$\sigma$-加法族$\Sigma'$に対して$\Sigma\subseteq\Sigma'$
\end{enumerate}
\end{prop}

\begin{proof}
$\mathcal{A}\subseteq\Sigma'$を満たす$S$上の$\sigma$-加法族$\Sigma'$全体の集合を$\mathcal{E}$とする.
このとき
\[ \Sigma:=\bigcap_{\Sigma'\in\mathcal{E}}\Sigma' \]
と定めれば,$\Sigma$は1,2をともに満たす$S$上の$\sigma$-加法族である.
\end{proof}

\begin{qst}\label{qst_proof_of_generated_sigma_algebra}
命題\ref{prop_generated_sigma_algebra}において,
$\Sigma$が1,2をともに満たし,かつ$S$上の$\sigma$-加法族であることを示せ.
\end{qst}

\begin{defi}\label{def_generated_sigma_algebra}
集合$S$の任意の部分集合族$\mathcal{A}$に対し,
$\mathcal{A}$を含む最小の$\sigma$-加法族を$\sigma(\mathcal{A})$と書き,
$\mathcal{A}$によって生成される$\sigma$-加法族と呼ぶ.
\end{defi}

\begin{defi}\label{def_Borel_set}
位相空間$S$に対し,$S$の開集合をすべて含む最小の$\sigma$-加法族を$\mathcal{B}(S)$と書く.
\end{defi}

\section{可測空間}

\begin{defi}\label{def_measurable_space}
集合$S$と$S$上の$\sigma$-加法族$\Sigma$の組$(S,\Sigma)$を\textbf{可測空間}と呼ぶ.
\end{defi}

\section{集合関数の有限加法性と$\sigma$-加法性}

\begin{defi}\label{def_additive}
集合$S$と$S$上の有限加法族$\Sigma_0$に対し,集合関数
$\mu_0\colon\Sigma_0\to[0,\infty]$
が\textbf{有限加法的}であるとは,次の条件をすべて満たすことである:
\begin{enumerate}
\item$\mu_0(\varnothing)=0$
\item$A_1,A_2\in\Sigma_0$に対して
$A_1 \cap A_2=\varnothing\Rightarrow\mu_0(A_1 \cup A_2)=\mu_0(A_1)+\mu_0(A_2)$
\end{enumerate}
\end{defi}

\begin{defi}\label{def_disjoint_union}
集合$A_1,A_2,\cdots$が,$i \neq j$に対して$A_i \cap A_j=\varnothing$を満たすとき,
\[ \sum_{n=1}^k A_n:=\bigcup_{n=1}^k A_n,\sum_{n=1}^\infty A_n:=\bigcup_{n=1}^\infty A_n \]
と定める.
\end{defi}

\begin{defi}\label{def_countably_additive}
集合$S$と$S$上の有限加法族$\Sigma_0$に対し,集合関数
$\mu_0\colon\Sigma_0\to[0,\infty]$
が\textbf{$\sigma$-加法的}であるとは,次の条件をすべて満たすことである:
\begin{enumerate}
\item$\mu_0(\varnothing)=0$
\item$\displaystyle\mu_0\left(\sum_{n=1}^\infty A_n\right)=\sum_{n=1}^\infty\mu_0(A_n)$
\end{enumerate}
\end{defi}

\begin{qst}\label{qst_countably_additive_then_additive}
集合$S$と$S$上の有限加法族$\Sigma_0$に対し,集合関数
$\mu_0\colon\Sigma_0\to[0,\infty]$
が$\sigma$-加法的であるとき,$\mu_0$は有限加法的であることを示せ.
\end{qst}

\section{測度}

\begin{defi}\label{def_measure}
可測空間$(S,\Sigma)$上の集合関数$\mu \colon \Sigma\to[0,\infty]$が\textbf{測度}であるとは,
$\mu$が$\sigma$-加法的であることをいう.
\end{defi}

\section{測度空間}

\begin{defi}\label{def_measure_space}
可測空間$(S,\Sigma)$および$(S,\Sigma)$上の測度$\mu \colon \Sigma\to[0,\infty]$に対して,
組$(S,\Sigma,\mu)$を\textbf{測度空間}と呼ぶ.
\end{defi}

\chapter{可測関数}

$(S,\Sigma)$を可測空間とする.

\section{可測関数の定義}

\begin{defi}\label{def_measurable_function}
$S$上の関数$h \colon S\to\mathbb{R}$が\textbf{$\Sigma$-可測関数}であるとは,
任意の$B\in\mathcal{B}(\mathbb{R})$に対して$h^{-1}(B)\in\Sigma$が成り立つことをいう.
\end{defi}

\begin{defi}\label{def_indicator_function}
$A\in\Sigma$に対し,$S$上の関数$1_A \colon S\to\mathbb{R}$を
\[ 1_A(s):=\begin{cases}1 & (s \in A) \\ 0 & (s \notin A)\end{cases} \]
により定める.このような関数を\textbf{定義関数}と呼ぶ.
\end{defi}

\begin{qst}\label{qst_indicator_function_is_measurable}
$A\in\Sigma$に対し,定義関数$1_A \colon S\to\mathbb{R}$は$\Sigma$-可測関数であることを示せ.
\end{qst}

\section{可測関数であることを示すテクニック}

問\ref{qst_indicator_function_is_measurable}のように簡単な関数であれば,
定義に従って可測関数であることを示すことができる.
しかし,定義に従って示すのは難しいことが多い.
本節では,そのような場合に役立つ命題を紹介する.

\begin{lem}\label{lem_measurability_and_generated_sigma_algebra}
$\mathcal{A}\subseteq\mathcal{B}(\mathbb{R})$が
$\sigma(\mathcal{A})=\mathcal{B}(\mathbb{R})$を満たすとする.
このとき,$S$上の関数$h \colon S\to\mathbb{R}$が,
任意の$A\in\mathcal{A}$に対して$h^{-1}(A)\in\Sigma$を満たすならば,
$h$は$\Sigma$-可測関数である.
\end{lem}

\begin{proof}
$\mathcal{E}:=\{B\in\mathcal{B}(\mathbb{R}):h^{-1}(B)\in\Sigma\}$とする.
このとき,$\mathcal{E}$は$\sigma$-加法族であり,
さらに仮定より$\mathcal{A}\subseteq\mathcal{E}$,
$\sigma(\mathcal{A})=\mathcal{B}(\mathbb{R})$である.
従って,$\mathcal{E}=\mathcal{B}(\mathbb{R})$が成り立つ.
\end{proof}

\begin{qst}\label{qst_proof_of_measurability_and_generated_sigma_algebra}
補題\ref{lem_measurability_and_generated_sigma_algebra}の証明において,次の事実を確かめよ:
\begin{enumerate}
\item$\mathcal{E}$は$\sigma$-加法族である.
\item$\mathcal{E}=\mathcal{B}(\mathbb{R})$が成り立つ.
\end{enumerate}
\end{qst}

\begin{lem}\label{lem_what_Borel_real_set_is_generated_by}
$\pi(\mathbb{R}):=\{(-\infty,a]:a\in\mathbb{R}\}$とする.
このとき$\mathcal{B}(\mathbb{R})=\sigma(\pi(\mathbb{R}))$である.
\end{lem}

\begin{proof}
任意の$a\in\mathbb{R}$に対して
$(-\infty,a]=\displaystyle\bigcap_{n=1}^\infty(-\infty,a+n^{-1})\in\mathcal{B}(\mathbb{R})$
である,すなわち$\pi(\mathbb{R})\subseteq\mathcal{B}(\mathbb{R})$が成り立ち,
$\mathcal{B}(\mathbb{R})$は$\sigma$-加法族であるから
$\sigma(\pi(\mathbb{R}))\subseteq\mathcal{B}(\mathbb{R})$となる.
また,任意の$a,b\in\mathbb{R}$に対して
\[ (a,b)=(a,\infty)\cap(-\infty,b)
=(-\infty,a]^c\cap\bigcup_{n=1}^\infty(-\infty,b-n^{-1}]\in\sigma(\pi(\mathbb{R})) \]
である.
ここで,$\mathbb{R}$の任意の開集合が開区間の可算和で表せることに注意すると,
$\mathbb{R}$の開集合はすべて$\sigma(\pi(\mathbb{R}))$の元となることがわかる.
従って,$\sigma(\pi(\mathbb{R}))$は$\sigma$-加法族であるから
$\mathcal{B}(\mathbb{R})\subseteq\sigma(\pi(\mathbb{R}))$となる.
\end{proof}

\begin{qst}\label{qst_real_open_set_is_countable_union_of_open_interval}
$\mathbb{R}$の任意の開集合$O$は開区間の可算和で表せることを証明せよ.
すなわち,ある実数列$\{a_n\}_{n=1}^\infty, \{b_n\}_{n=1}^\infty$が存在して,
$O=\displaystyle\bigcup_{n=1}^\infty(a_n,b_n)$となることを示せ.
\end{qst}

補題\ref{lem_measurability_and_generated_sigma_algebra}
および補題\ref{lem_what_Borel_real_set_is_generated_by}から,次の命題が従う.

\begin{prop}\label{prop_technique_to_check_measurability}
$S$上の関数$h \colon S\to\mathbb{R}$が,
任意の$a\in\mathbb{R}$に対して$\{h \leq a\}\in\Sigma$を満たすならば,
$h$は$\Sigma$-可測関数である.
\end{prop}

\begin{qst}\label{qst_technique_to_check_measurability}
$S$上の関数$h \colon S\to\mathbb{R}$が次のいずれかの条件を満たすならば,
$h$は$\Sigma$-可測関数であることを示せ.
\begin{enumerate}
\item 任意の$a\in\mathbb{R}$に対して$\{h > a\}\in\Sigma$
\item 任意の$a\in\mathbb{R}$に対して$\{h < a\}\in\Sigma$
\item 任意の$a\in\mathbb{R}$に対して$\{h \geq a\}\in\Sigma$
\end{enumerate}
\end{qst}

\section{可測関数の性質}

\begin{prop}\label{prop_measurable_function_and_arithmetic}
$\Sigma$-可測関数$h_1,h_2 \colon S\to\mathbb{R}$および$a\in\mathbb{R}$に対し,
$h_1+h_2, ah_1, h_1h_2$は$\Sigma$-可測関数である.
\end{prop}

\begin{proof}
$b\in\mathbb{R}$および$s \in S$に対し,$h_1(s)+h_2(s)>b$であることは,
ある$q\in\mathbb{Q}$が存在して$h_1(s)>q>b-h_2(s)$が成り立つことと同値である.
従って,$h_1,h_2$は$\Sigma$-可測なので
\[ \{h_1+h_2>b\}=\bigcup_{q\in\mathbb{Q}}\{h_1>q>b-h_2\}
=\bigcup_{q\in\mathbb{Q}}(\{h_1>q\}\cap\{h_2>b-q\})\in\Sigma \]
となり,$h_1+h_2$が$\Sigma$-可測であることが従う.
$ah_1$について,$a=0$のときは定数関数ゆえ$\Sigma$-可測である.
$a>0$のとき,$b\in\mathbb{R}$に対して$\{ah_1>b\}=\{h_1>a^{-1}b\}\in\Sigma$,
$a<0$のとき,$b\in\mathbb{R}$に対して$\{ah_1>b\}=\{h_1<a^{-1}b\}\in\Sigma$
となり,いずれの場合も$ah_1$は$\Sigma$-可測である.
$h_1h_2$について,$h_1=h_2$のとき,
$b<0$ならば$\{h_1h_2>b\}=\{h_1^2>b\}=S\in\Sigma$,
$b\geq0$ならば$\{h_1h_2>b\}=\{h_1^2>b\}=\{h_1<-\sqrt{b}\}\cup\{h_1>\sqrt{b}\}\in\Sigma$
となるから,$h_1h_2$は$\Sigma$-可測である.
$h_1 \neq h_2$のときは,$h_1h_2=4^{-1}\{(h_1+h_2)^2-(h_1-h_2)^2\}$であることに注意すれば,
これまでの議論から$h_1h_2$が$\Sigma$-可測となることがわかる.
\end{proof}

\begin{defi}\label{def_positive_and_negative_part}
関数$f \colon S\to\mathbb{R}$に対して
$f^+(s):=\max\{f(s),0\}, f^-(s):=\max\{-f(s),0\}$
と定める.
\end{defi}

\begin{qst}\label{qst_positive_and_negative_part_are_measurable}
$\Sigma$-可測関数$f \colon S\to\mathbb{R}$に対して,$f^+,f^-$は非負$\Sigma$-可測関数であることを示せ.
\end{qst}

\begin{qst}\label{qst_absolute_is_measurable}
$\Sigma$-可測関数$f \colon S\to\mathbb{R}$に対して,$|f|$は非負$\Sigma$-可測関数であることを示せ.
\end{qst}

\begin{defi}\label{def_simple_function}
$a_1,a_2,\cdots,a_n\in\mathbb{R}$および$A_1,A_2,\cdots,A_n\in\Sigma$に対し,
\[ \sum_{i=1}^n a_i1_{A_i}(s) \]
の形で表せる$S$上の関数を\textbf{単関数}と呼ぶ.
\end{defi}

もちろん,単関数は可測関数である.

\begin{prop}\label{prop_properties_of_nonnegative_simple_function}
非負単関数$f,g \colon S\to[0,\infty)$と$a\geq0$に対して,$f+g, af$は非負単関数である.
また,$(f \vee g)(s):=\max\{f(s),g(s)\}, (f \wedge g)(s):=\min\{f(s),g(s)\}$も非負単関数である.
\end{prop}

\begin{qst}\label{qst_proof_of_properties_of_nonnegative_simple_function}
命題\ref{prop_properties_of_nonnegative_simple_function}を証明せよ.
\end{qst}

\chapter{積分}

$(S,\Sigma,\mu)$を測度空間とする.

\section{定義関数の積分}

\begin{defi}\label{def_indicator_function_integral}
$A\in\Sigma$に対し,定義関数$1_A \colon S\to\mathbb{R}$の積分を
\[ \int_S 1_A(s)\mu(ds):=\mu(A) \]
で定める.
\end{defi}

\section{非負単関数の積分}

\begin{defi}\label{def_nonnegative_simple_function_integral}
$a_1,a_2,\cdots,a_n\in[0,\infty)$および$A_1,A_2,\cdots,A_n\in\Sigma$に対し,
非負単関数$f(s)=\displaystyle\sum_{i=1}^n a_i1_{A_i}(s)$の積分を
\[ \int_S f(s)\mu(ds):=\sum_{i=1}^n a_i\mu(A_i) \]
で定める.
\end{defi}

\begin{qst}\label{qst_simple_function_integral}
定義\ref{def_nonnegative_simple_function_integral}はwell-definedであることを示せ.
すなわち,非負単関数$f \colon S\to[0,\infty)$が
\[ f(s)=\sum_{i=1}^n a_i1_{A_i}(s)=\sum_{i=1}^m b_i1_{B_i}(s) \]
の2通りで表されるとき,
\[ \sum_{i=1}^n a_i\mu(A_i)=\sum_{i=1}^m b_i\mu(B_i) \]
となることを示せ.
\end{qst}

\begin{prop}\label{prop_properties_of_simple_function_integral}
非負単関数$f,g \colon S\to[0,\infty)$と$a\geq0$に対して,次の性質が成り立つ:
\begin{enumerate}
\item$\mu(f \neq g)=0\Rightarrow\displaystyle\int_S f(s)\mu(ds)=\int_S g(s)\mu(ds)$
\item$\displaystyle\int_S \{f(s)+g(s)\}\mu(ds)=\int_S f(s)\mu(ds)+\int_S g(s)\mu(ds),
\int_S af(s)\mu(ds)=a\int_S f(s)\mu(ds)$
\item$f \leq g\Rightarrow\displaystyle\int_S f(s)\mu(ds)\leq\int_S g(s)\mu(ds)$
\end{enumerate}
\end{prop}

\begin{qst}\label{qst_proof_of_properties_of_simple_function_integral}
命題\ref{prop_properties_of_simple_function_integral}を証明せよ.
\end{qst}

\section{非負可測関数の積分}

\begin{defi}\label{def_nonnegative_function_integral}
非負$\Sigma$-可測関数$f \colon S\to\mathbb[0,\infty)$の積分を
\[ \int_S f(s)\mu(ds):=\sup\left\{\int_Sh(s)\mu(ds):hは非負単関数でh \leq fを満たす\right\} \]
で定める.
\end{defi}

\begin{qst}\label{qst_nonnegative_function_integral_and_simple_function}
非負単関数$f$に対して,定義\ref{def_nonnegative_function_integral}の等号が成り立つことを示せ.
\end{qst}

\begin{thm}[\textbf{単調収束定理}]\label{monotone_convergence_theorem}
非負$\Sigma$-可測関数列$\{f_n\}_{n=1}^\infty$と$\Sigma$-可測関数$f$が,
すべての$s \in S$に対して$0 \leq f_1(s) \leq f_2(s) \leq\cdots$かつ
$\displaystyle\lim_{n\to\infty}f_n(s)=f(s)$を満たすとき,次が成り立つ:
\[ \lim_{n\to\infty}\int_S f_n(s)\mu(ds)=\int_S f(s)\mu(ds) \]
\end{thm}

\begin{prop}\label{prop_linearity_of_nonnegative_function_integral}
$a,b\in[0,\infty)$と非負$\Sigma$-可測関数$f,g$に対して次が成り立つ:
\[ \int_S\{af(s)+bg(s)\}\mu(ds)=a\int_Sf(s)\mu(ds)+b\int_Sg(s)\mu(ds) \]
\end{prop}

\section{可測関数の積分}

\begin{defi}\label{def_integrable_function}
$\Sigma$-可測関数$f \colon S\to\mathbb{R}$が$\mu$について\textbf{可積分}であるとは,
$\displaystyle\int_S|f(s)|\mu(ds)<\infty$であることをいう.

\end{defi}

\begin{defi}\label{def_measurable_function_integral}
可積分関数$f \colon S\to\mathbb{R}$の積分を
\[ \int_S f(s)\mu(ds):=\int_S f^+(s)\mu(ds)-\int_S f^-(s)\mu(ds)  \]
で定める.
\end{defi}

\begin{qst}\label{qst_absolute_integral_and_integral_absolute}
可積分関数$f$に対して
$\displaystyle\left|\int_Sf(s)\mu(ds)\right|\leq\int_S|f(s)|\mu(ds)$
が成り立つことを示せ.
\end{qst}

\begin{prop}\label{prop_linearity_of_integrable_function_integral}
$a,b\in\mathbb{R}$と可積分関数$f,g$に対して,
$af+bg$は可積分関数であり,次が成り立つ:
\[ \int_S\{af(s)+bg(s)\}\mu(ds)=a\int_Sf(s)\mu(ds)+b\int_Sg(s)\mu(ds) \]
\end{prop}

\section{優収束定理}

\begin{lem}[\textbf{ファトゥの補題}]\label{Fatou_lemma}
非負$\Sigma$-可測関数列$\{f_n\}_{n=1}^\infty$に対して次が成り立つ:
\[ \int_S\liminf_{n\to\infty}f_n(s)\mu(ds)\leq\liminf_{n\to\infty}\int_S f_n(s)\mu(ds) \]
\end{lem}

\begin{qst}\label{qst_proof_of_Fatou_lemma}
$g_n(s):=\displaystyle\inf_{m \geq n}f_m(s)$とする.関数列$\{g_n\}_{n=1}^\infty$に
定理\ref{monotone_convergence_theorem}を適用して,補題\ref{Fatou_lemma}を証明せよ.
\end{qst}

\begin{qst}\label{qst_counterexample_of_Fatou_lemma}
補題\ref{Fatou_lemma}の反例を挙げよ.
\end{qst}

\begin{cor}\label{reverse_Fatou_lemma}
非負$\Sigma$-可測関数列$\{f_n\}_{n=1}^\infty$に対して,
ある非負$\Sigma$-可測関数$g$が存在して
$f_n \leq g\ (n=1,2,\cdots)$かつ$\mu(g)<\infty$を満たすとき,
次が成り立つ:
\[ \int_S\limsup_{n\to\infty}f_n(s)\mu(ds)\geq\limsup_{n\to\infty}\int_S f_n(s)\mu(ds) \]
\end{cor}

\begin{thm}[\textbf{優収束定理}]\label{dominated_convergence_theorem}
$\Sigma$-可測関数列$\{f_n\}_{n=1}^\infty$と$\Sigma$-可測関数$f$が,
すべての$s \in S$に対して$\displaystyle\lim_{n\to\infty}f_n(s)=f(s)$を満たすとする.
さらに,ある非負可積分関数$g$が存在して,$\displaystyle\int_Sg(s)\mu(ds)<\infty$かつ
すべての$n=1,2,\cdots$および$s \in S$に対して$|f_n(s)| \leq g(s)$を満たすとする.
このとき,次が成り立つ:
\[ \lim_{n\to\infty}\int_Sf_n(s)\mu(ds)=\int_Sf(s)\mu(ds) \]
\end{thm}

% \part{$\mathbb{R}$上のルベーグ積分}
% \chapter{1次元ルベーグ測度}
% \chapter{ルベーグ可測空間}
% \chapter{可測関数}
% \chapter{積分}

% 参考文献
\begin{thebibliography}{9}
\item David Williams, \textit{Probability with Martingales}, Cambridge University Press, 1991
\end{thebibliography}

\end{document}